% !TEX program = xelatex
% !TeX encoding = UTF-8
\documentclass[10pt]{beamer}

\usetheme[progressbar=frametitle,sectionpage=progressbar]{metropolis}
\usepackage{appendixnumberbeamer}

\usepackage{booktabs}
\usepackage[scale=2]{ccicons}
\usepackage{xeCJK}
\usepackage{graphicx}
\usepackage{pgfplots}
\usepgfplotslibrary{dateplot}

\usepackage{xspace}
\newcommand{\themename}{\textbf{\textsc{metropolis}}\xspace}

\title{华中科技大学幻灯片模板}
\subtitle{DEMO}
\date{\today}
\author{Percy Liu}
\institute{华中科技大学}
% \titlegraphic{\hfill\includegraphics[height=1.5cm]{logo.pdf}}

\begin{document}

\maketitle

\begin{frame}{超长标题测试超长标题测试超长标题测试}
  \setbeamertemplate{section in toc}[sections numbered]
  \tableofcontents[hideallsubsections]
\end{frame}

\section{引言}

\begin{frame}[fragile]{扁平化风格模板}

  该模板基于 \themename,通过扁平化字体、色块与简化背景,最大程度降低幻灯片中不必要的元素,旨在提供一种简约阅读体验.

  使用该主题的方法:
仅需在文件头加入
  \begin{verbatim}    \documentclass{beamer}
    \usetheme{metropolis}\end{verbatim}

  注:幻灯片字体采用FireFox公司的Fira Sans字体。
\end{frame}
\begin{frame}[fragile]{章节测试}
  每个章节集中讨论一个话题,方便读者理解幻灯片内容

  \begin{verbatim}    元素化章节 \end{verbatim}

  同时,提供了简约式进度条,提示读者当前阅读位置。
\end{frame}

\section{标题格式测试}

\begin{frame}{标题格式测试}
	英文模板支持下列四种标题格式
	\begin{itemize}
		\item Regular 正体
		\item \textsc{Small caps} 大写开头
		\item \textsc{all small caps}全小写
		\item ALL CAPS 全大写
	\end{itemize}
	每个章节可单独设置标题格式
\end{frame}

\section{基本功能测试}

\begin{frame}[fragile]{字形测试}
      \begin{verbatim}幻灯片主题包括强调、加粗、斜体等多种默认字形命令\end{verbatim}

  \begin{center}居中\end{center}

 示例:\emph{强调}, \alert{警告} 和 \textbf{加粗}.
\end{frame}

\begin{frame}{字体功能性测试}
  \begin{itemize}
    \item 正体
    \item \textit{加斜}
    \item \textsc{Small Caps}
    \item \textbf{加粗}
    \item \textbf{\textit{加粗加斜}}
    \item \textbf{\textsc{Bold Small Caps}}
  \end{itemize}
\end{frame}

\begin{frame}{列表}
  \begin{columns}[T,onlytextwidth]
    \column{0.33\textwidth}
      列举
      \begin{itemize}
        \item 牛奶 \item 鸡蛋 \item 土豆
      \end{itemize}

    \column{0.33\textwidth}
      枚举
      \begin{enumerate}
        \item 第一, \item 第二, \item 最后.
      \end{enumerate}

    \column{0.33\textwidth}
      描述
      \begin{description}
        \item[PowerPoint] 幻灯片. \item[Beamer] 正片.
      \end{description}
  \end{columns}
\end{frame}
\begin{frame}{动画}
  \begin{itemize}[<+- | alert@+>]
    \item \alert<4>{这是重点}
    \item 接下来
    \item 然后
  \end{itemize}
\end{frame}
\begin{frame}{图片测试}
  \begin{figure}
    \newcounter{density}
    \setcounter{density}{20}
    \begin{tikzpicture}
      \def\couleur{alerted text.fg}
      \path[coordinate] (0,0)  coordinate(A)
                  ++( 90:5cm) coordinate(B)
                  ++(0:5cm) coordinate(C)
                  ++(-90:5cm) coordinate(D);
      \draw[fill=\couleur!\thedensity] (A) -- (B) -- (C) --(D) -- cycle;
      \foreach \x in {1,...,40}{%
          \pgfmathsetcounter{density}{\thedensity+20}
          \setcounter{density}{\thedensity}
          \path[coordinate] coordinate(X) at (A){};
          \path[coordinate] (A) -- (B) coordinate[pos=.10](A)
                              -- (C) coordinate[pos=.10](B)
                              -- (D) coordinate[pos=.10](C)
                              -- (X) coordinate[pos=.10](D);
          \draw[fill=\couleur!\thedensity] (A)--(B)--(C)-- (D) -- cycle;
      }
    \end{tikzpicture}
    \caption{方形旋转
    \href{http://www.texample.net/tikz/examples/rotated-polygons/}{texample.net}.}
  \end{figure}
\end{frame}
\begin{frame}{表格测试}
  \begin{table}
    \caption{全球最大的城市 (来源: 维基百科)}
    \begin{tabular}{@{} lr @{}}
      \toprule
      城市 & 人口\\
      \midrule
      墨西哥城 & 20,116,842\\
      上海 & 19,210,000\\
      北京 & 15,796,450\\
      伊斯坦布尔 & 14,160,467\\
      \bottomrule
    \end{tabular}
  \end{table}
\end{frame}
\begin{frame}{字块与色块测试}
  不同色块与不同色彩的字块可任意进行搭配

  \begin{columns}[T,onlytextwidth]
    \column{0.5\textwidth}
      \begin{block}{默认}
        内容块
      \end{block}

      \begin{alertblock}{警告}
        内容块
      \end{alertblock}

      \begin{exampleblock}{例子}
        内容块
      \end{exampleblock}

    \column{0.5\textwidth}

      \metroset{block=fill}

      \begin{block}{默认}
        内容块
      \end{block}

      \begin{alertblock}{警告}
        内容块
      \end{alertblock}

      \begin{exampleblock}{例子}
        内容块
      \end{exampleblock}

  \end{columns}
\end{frame}
\begin{frame}{数学环境测试}
  \begin{equation*}
    e = \lim_{n\to \infty} \left(1 + \frac{1}{n}\right)^n
  \end{equation*}
\end{frame}
\begin{frame}{线型图}
  \begin{figure}
    \begin{tikzpicture}
      \begin{axis}[
        mlineplot,
        width=0.9\textwidth,
        height=6cm,
      ]

        \addplot {sin(deg(x))};
        \addplot+[samples=100] {sin(deg(2*x))};

      \end{axis}
    \end{tikzpicture}
  \end{figure}
\end{frame}
\begin{frame}{柱状图}
  \begin{figure}
    \begin{tikzpicture}
      \begin{axis}[
        mbarplot,
        xlabel={项目},
        ylabel={高度},
        width=0.9\textwidth,
        height=6cm,
      ]

      \addplot plot coordinates {(1, 20) (2, 25) (3, 22.4) (4, 12.4)};
      \addplot plot coordinates {(1, 18) (2, 24) (3, 23.5) (4, 13.2)};
      \addplot plot coordinates {(1, 10) (2, 19) (3, 25) (4, 15.2)};

      \legend{蓝色, 黄色, 绿色}

      \end{axis}
    \end{tikzpicture}
  \end{figure}
\end{frame}
\begin{frame}{引用}
  \begin{quote}
    Veni, Vidi, Vici
  \end{quote}
\end{frame}

{%
\setbeamertemplate{脚标}{My custom footer}
\begin{frame}[fragile]{Frame footer}
    本模板定义了一种美观的脚标。使用方法如下:
    \begin{verbatim}\setbeamertemplate{frame footer}{自定义脚标}\end{verbatim}
\end{frame}
}

\begin{frame}{参考文献}
  本文应用了部分参考文献 \cite{knuth92,ConcreteMath,Simpson,Er01,greenwade93}
\end{frame}

\section{小结}

\begin{frame}{总结}

  本模板基于\themename

  \begin{center}\url{github.com/matze/mtheme}\end{center}。

 因此,本模板 \emph{itself} 遵循CCAS 4.0
  \href{http://creativecommons.org/licenses/by-sa/4.0/}{Creative Commons
  Attribution-ShareAlike 4.0 International License}.

  \begin{center}\ccbysa\end{center}

\end{frame}

\begin{frame}[standout]
   \Huge{谢谢观赏}
\end{frame}

\appendix

\begin{frame}[fragile]{备用}
  备用幻灯片可以帮助读者更细腻的了解展示内容。

  仅需在幻灯片开头包含 \verb|appendixnumberbeamer|
  包 并在附录页面开始前声明\verb|\appendix|。

  本模板会自动关闭附录中的页码和标号功能。
\end{frame}

\begin{frame}[allowframebreaks]{References}

  \bibliography{demo}
  \bibliographystyle{abbrv}

\end{frame}

\end{document}
